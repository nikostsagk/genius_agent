% This is samplepaper.tex, a sample chapter demonstrating the
% LLNCS macro package for Springer Computer Science proceedings;
% Version 2.20 of 2017/10/04
%
\documentclass[runningheads]{llncs}
%
\usepackage{graphicx}
\usepackage[toc,page]{appendix}
\usepackage{listings}
\usepackage{color}
\usepackage{float}
\lstset{
  basicstyle=\ttfamily,
  columns=fullflexible,
%  frame=single,
  breaklines=true,
}

% Used for displaying a sample figure. If possible, figure files should
% be included in EPS format.
%
% If you use the hyperref package, please uncomment the following line
% to display URLs in blue roman font according to Springer's eBook style:
% \renewcommand\UrlFont{\color{blue}\rmfamily}

\begin{document}
%
\title{Intelligent Agents}
%
%\titlerunning{Abbreviated paper title}
% If the paper title is too long for the running head, you can set
% an abbreviated paper title here
%
\author{Dimitri Diomaiuta (30598109) - Group 9}
%
% \authorrunning{F. Author et al.}
% First names are abbreviated in the running head.
% If there are more than two authors, 'et al.' is used.
%
\institute{University of Southampton}
%% \institute{Princeton University, Princeton NJ 08544, USA \and
%% Springer Heidelberg, Tiergartenstr. 17, 69121 Heidelberg, Germany
%% \email{lncs@springer.com}\\
%% \url{http://www.springer.com/gp/computer-science/lncs} \and
%% ABC Institute, Rupert-Karls-University Heidelberg, Heidelberg, Germany\\
%% \email{\{abc,lncs\}@uni-heidelberg.de}}
%
\maketitle              % typeset the header of the contribution
%
\begin{abstract}
In this paper we analyze the design and implementation of a
negotiation agent built on the \textit{GENIUS} framework \cite{genius}.
\end{abstract}
%
%
%

\section{Design}
In this section we explain the design choices we made in constructing
our agent. As per this year ANAC competition \cite{anac}, the agent
has to be designed to work with uncertainty and discrete values. In
the following subsections we explain the different aspects of our
design.

\subsection{Uncertainty}
When an agent works with uncertainty it means that does not have full
knowledge of its utility space. Instead, the agent has a list of bids
sorted from the worst to the best one. It is fundamental to know our
utility space in order to be able to bid on the preferred items and
obtain a successful negotiation result. To solve this issue, we designed
our agent to model its utility space by analyzing the order of the
preferred bids.

\subsection{Concession strategy}\label{concession_strategy}
A concession strategy is needed in order to reach an agreement with
the opponent. The idea is that if both agents start bidding their best
options there is a high chance that an agreement is never found
if they do not renounce to maximizing their utility. A concession
strategy is all about renouncing to maximize our utility and, hence,
offer bids that are not at the top of our agent's list. We opted for a time dependent concession
strategy. This means that our agent starts conceding more as time
passes. We can see the mathematical notation of what we have just
described at \ref{conc}.
\begin{equation}\label{conc}
  utility = 1 - time^{1/\beta}
\end{equation}
Figure \ref{figconc} represents the different concession strategy based
on different $\beta$ value.
\begin{figure}
\includegraphics[width=\textwidth]{./img/boul2.png}
\caption{Different $\beta$ values for the concession strategy. Utility
at x-axis and time at y-axis.} \label{figconc}
\end{figure}
We designed our agent to play a Boulware concession strategy with
$\beta = 0.3$. As we can observe from figure \ref{figconc} we can see
that this strategy provides the flexibility of utility slow descent
until reaching a low enough utility to find an agreement. 

\subsection{Opponent modelling}\label{opmod}
Another important feature of our strategy is opponent modelling. The
breakthrough of this idea is that if our agent knows what the opponent
wants we can find a bidding that maximizes both agents utility
and, hence, have an higher probability to be closer to both the
Pareto frontier and the Nash point. We apply this technique by
exploiting the Johnny Black strategy \cite{black}. The main idea is
that the options the opponent bids the most are also a good measure of
the opponent's preferences. In order to model these preferences we
construct a frequency table where we store the opponent's bidding
history. Every time our agent receives an offer, the frequency
table is updated. When the agent gathers enough information about the
opponent, it updates its 
model by first ordering the options of the issues following
equation \ref{gd}. $k$ is the number of possible options for the
given issue and $n_o$ the rank of the current option $o$.
\begin{equation}\label{gd}
  V_0 = \frac{k - n_0 + 1}{k}
\end{equation}
The weights of the issues are then given by equation \ref{dd}. $f_o$ and $t$ represent respectively the frequency of option $o$ and
the total number of opponent's previous bids.
\begin{equation}\label{dd}
  \hat{w_i} = \sum_{o \in O_i} \frac{f_o^2}{t^2}
\end{equation}
The issues weights are then normalized following equation \ref{norm}.
\begin{equation}\label{norm}
w_i = \frac{\hat{w_i}}{\sum_{j \in I} \hat{w_j}}
\end{equation}

\subsection{Offer strategy}
Our agent's offer strategy is based on the opponent modelling and on
the concession strategy. The utility space is searched in order to
find all the bids within a utility range. The upper bound is $1$, the
maximum possible utility, while the lower bound is given by the result
of the concession strategy equation \ref{conc}. The estimation of the
opponent utility is described in \ref{opmod}. From all the obtained
bids we offer to the opponent the one that has the best joint utility for
our agent and for the opponent as well, see equation \ref{joint}.
\begin{equation}\label{joint}
  joint_{utility} = agent_{utility} * opponent_{utility}
\end{equation}
This strategy allows our agent to offer bids that are close to the
Pareto frontier and to the Nash point. 

\subsection{Acceptance strategy}
Our agent's acceptance strategy is consistent with the offer one. We
accept the opponent's offer if it meets one of the following
requirements:
\begin{itemize}
\item The joint utility is bigger than our agent's last offer joint
utility.
\item Our agent utility is bigger than our agent's last offer utility.
\end{itemize}
This strategy allows to end the negotiation with a reasonable
agreement, avoiding the utility descent caused by the concession
strategy.

\section{Results}\label{results}
In this section we analyze the results of our Agent. The results have
been gathered during the agents competition organized by the course
\textit{Intelligent Agents (COMP6203)} of \textit{University of
  Southampton}. In this competition all the implemented agents for the
course have been tested against each other on three different domain
sizes: small, medium and large. The metrics of evaluation were the
agent's utility, distance to Nash point and agreement rate. We can see the results
summarized in tables \ref{tab_agreement}, \ref{tab_utility} and
\ref{tab_nash}, showing our agent, \textit{Agent 9}, and all other
agents outcome.
\begin{table}[H]
\caption{The agreement rate results.}\label{tab_agreement}
\begin{tabular}{|l|r|r|r|}
\hline
\textbf{Agents}~~~ & \textbf{Small Domain}~~~ &
\textbf{Medium Domain}~~~ & \textbf{Large Domain}~~~\\
\hline
Agent 9 & 96.39\% & 59.36\% & 71.98\% \\
\hline
All agents & 95.49\% & 72.25\% & 82.07\% \\
\hline
\end{tabular}
\end{table}

\begin{table}[H]
\caption{The average utility results.}\label{tab_utility}
\begin{tabular}{|l|r|r|r|}
\hline
\textbf{Agents}~~~ & \textbf{Small Domain}~~~ &
\textbf{Medium Domain}~~~ & \textbf{Large Domain}~~~\\
\hline
Agent 9 & 0.946 & 0.661 & 0.791 \\
\hline
All agents & 0.685 & 0.648 & 0.654 \\
\hline
\end{tabular}
\end{table}

\begin{table}[H]
\caption{The average distance to Nash point results.}\label{tab_nash}
\begin{tabular}{|l|r|r|r|}
\hline
\textbf{Agents}~~~ & \textbf{Small Domain}~~~ &
\textbf{Medium Domain}~~~ & \textbf{Large Domain}~~~\\
\hline
Agent 9 & 0.070 & 0.510 & 0.334 \\
\hline
All agents & 0.147 & 0.763 & 0.552 \\
\hline
\end{tabular}
\end{table}

We can observe that, regarding the agreement rate (see table \ref{tab_agreement}), our agent
performs well for the small domains but below average
in the medium and and large domains. This is probably because our
agent's Boulware strategy is too strong. Having a strong time
dependent concession strategy, as
described in subsection \ref{concession_strategy}, causes a slow
utility descent. In medium and large domains the increased number of
possible bids requires a less strong concession strategy that
facilitates utility descent and, hence, the possibility to find an
agreement with the opponent.

The average utility data, see table \ref{tab_utility}, suggest that
our agent's strong ability is indeed finding an agreement with an high
utility. In all the three domains our agent performs better than the
average of the other agents. We can also observe that it performs
extremely well for the small domains and that the average utility is
always bigger than 0.5.

Our agent, with respect to the distance to Nash point (see table
\ref{tab_nash}), performs better than the average of the other
agents. The results, though, are not entirely satisfactory for the
medium and large size domains. An average Nash point of $0.510$ and
$0.334$ means that the possible offer bids have not been adequately
searched and that the last proposed offer, or the accepted one, was
mistakenly recognized as acceptable.

We can see, throughout all the data, a general pattern in our agent's
behaviour. It performs incredibly well on small domains and, in
contrast, more poorly on medium and large domains. This is not only
dependent on the concession strategy but on the opponent modelling as
well. In medium and large domains, in fact, the utility space is much
bigger and the Johnny Black opponent modelling, see section
\ref{opmod}, fails in producing an accurate model. A weak
knowledge of opponents preferences causes the agent to bid or accept
offers that are interpreted with wrong joint probability results (see
\ref{joint}).

\section{Conclusion}
In this paper we have addressed the design of an intelligent agent
based on the \textit{GENIUS} framework \cite{genius}. We have analyzed
our agent's behaviour regarding the agreement rate, average utility and
average distance to Nash point metrics (section \ref{results}). We have
also observed, given the results data, some limitations of our
design. In order to improve our agent, future work should consider implementing
an adaptive concession strategy based on the domain size and a better
opponent modelling for medium and large domains.


\begin{thebibliography}{8}

\bibitem{genius}
Ii.tudelft.nl. (2019). Genius. [online] Available at:
http://ii.tudelft.nl/genius/ [Accessed 2 Jan. 2019].
  
\bibitem{anac}
Mmi.tudelft.nl. (2019). Automated Negotiating Agents Competition
(ANAC) - Negotiation. [online] Available at:
\url{http://mmi.tudelft.nl/negotiation/index.php/Automated\_Negotiating\_Agents\_Competition\_(ANAC)}
    [Accessed 2 Jan. 2019].

\bibitem{black}
Yucel, O., Hoffman, J. and Sen, S., 2017. Jonny Black: A Mediating Approach to Multilateral Negotiations. In Modern Approaches to Agent-based Complex Automated Negotiation (pp. 231-238). Springer, Cham.

\end{thebibliography}

\end{document}
